% Originele statuten van Het Vrije Podium
\documentclass[a4paper,10pt]{article}
\usepackage[dutch]{babel}
\usepackage{amsthm}
\usepackage{graphicx}

\theoremstyle{definition}
\newtheorem{titel}{\large Titel}
\newtheorem{artikel}{\large Artikel}
\renewcommand{\familydefault}{\sfdefault}

\newcommand{\ttext}[1]{\large \textbf{#1} \normalsize}
\newcommand{\hnl}{\hfill\newline}

\title{\vspace{-2.0cm}\includegraphics[scale=0.3]{logo.png}\\
  \vspace{1cm}Statuten Het Vrije Podium\\Ontwerpdocument}
\author{Dennis D'Ooghe\\Voorzitter 2014-2015}
\date{Versie 26/03/2015} % Vul datum in als je dit in de header wilt

\begin{document}
\maketitle

% ---------- DE KRING ----------

\begin{titel}\ttext{Basis van de kring}

  \begin{artikel}\hnl
    De vereniging draagt de naam "Het Vrije Podium", afgekort "HVP".
  \end{artikel}

  \begin{artikel}\hnl
    De vereniging onderschrijft het principe van het Vrij Onderzoek.
  \end{artikel}

  \begin{artikel}\hnl
    De vereniging erkent en onderschrijft de Universele verklaring van de rechten van de mens en het Europees verdrag tot bescherming van de rechten van de mens en de fundamentele vrijheden, ondertekend te Rome op 4 november 1950: Titel I (art 1 tot en met 18)(B.S. 19/08/1955).
  \end{artikel}

  \begin{artikel}\hnl
    De vereniging is opgericht in september 2014 voor onbepaalde duur.
    De stichters zijn Dennis D'Ooghe en Valerie Lever.
  \end{artikel}

  \begin{artikel}\hnl
    De vereniging is gevestigd op Campus Etterbeek van de Vrije Universiteit Brussel, Pleinlaan 2, 1050 Elsene.
  \end{artikel}

  \begin{artikel}\hnl
    Het doel van de vereniging is...
  \end{artikel}

\end{titel}

% --------- DE LEDEN ----------

\begin{titel}\ttext{De leden}

  \begin{artikel}\hnl
    Effectieve leden zijn leden die tot de doelgroep van de vereniging behoren en hun lidgeld hebben betaald.
    De vereniging telt minimum 10 effectieve leden die ingeschreven zijn aan de VUB.
  \end{artikel}

\end{titel}

% --------- DE ALGEMENE VERGADERING ----------

\begin{titel}\ttext{De algemene vergadering}

  \begin{artikel}\hnl
    De algemene vergadering is samengesteld uit alle effectieve leden.
    Zij wordt voorgezeten door de Voorzitter van de vereniging en komt minstens 1 maal per jaar samen.
  \end{artikel}

  \begin{artikel}\hnl
    De algemene vergadering beslist over de goedkeuring van het jaarverslag van het uittredend bestuur.
    Op de algemene vergadering worden tevens kandidaten bestuursleden voorgesteld.
    De algemene vergadering is bevoegd de statuten van de vereniging te wijzigen volgens de procedure vermeld in art. 7.
  \end{artikel}

  \begin{artikel}\hnl
    De algemene vergadering wordt samengeroepen door de voorzitter van de vereniging.
    Een buitengewone algemene vergadering moet worden samengeroepen na schriftelijke aanvraag van 1/2 van de bestuursleden van de vereniging of na schriftelijke aanvraag van 1/4 der effectieve leden van de vereniging.
    De leden moeten minstens 14 dagen op voorhand verwittigd worden via elektronische weg.
    Deze bijeenroeping moet een gedetailleerde dagorde bevatten.
  \end{artikel}

  \begin{artikel}\hnl
    De algemene vergadering kan slechts geldig vergaderen indien 1/10 van de effectieve leden aanwezig is.
    Indien dit quorum niet bereikt wordt, is de volgende algemene vergadering geldig, ongeacht het aantal aanwezige leden.
    Tussen beide algemene vergaderingen moet minstens een tijdspanne van 15 dagen liggen
  \end{artikel}

  \begin{artikel}\hnl
    De beslissingen worden genomen bij gewone meerderheid van de aanwezige stemmen.
    Bij wijziging van de statuten is een 2/3 meerderheid van de aanwezige stemmen vereist en dient ten minste 2/3 van de effectieve leden aanwezig te zijn.
  \end{artikel}

  \begin{artikel}\hnl
    De algemene vergadering kan slechts geldig vergaderen over punten die in de dagorde voorkomen.
    Elk lid kan een punt op de dagorde laten verschijnen mits dit ten minste 8 dagen voor de vergadering schriftelijk bij de voorzitter van de vereniging wordt ingediend.
  \end{artikel}

\end{titel}

% ---------- HET BESTUUR ----------

\begin{titel}\ttext{Het bestuur}

  \begin{artikel}\hnl
    Alleen effectieve leden zijn verkiesbaar in de organen van de vereniging.
  \end{artikel}

  \begin{artikel}\hnl
    \begin{enumerate}
      \item Het bestuur telt minstens 4 hoofdfuncties:
      \begin{enumerate}
        \item De voorzitter zit de bestuursvergaderingen en de algemene vergadering voor, organiseert de algemene vergadering, vertegenwoordigt de vereniging op allerhande gelegenheden, organiseert allerhande activiteiten en staat in voor de verkiezing van het nieuw bestuur.
        \item De ondervoorzitter vervangt de voorzitter bij diens afwezigheid.
        \item De secretaris maakt een verslag van alle bestuursvergaderingen, van alle activiteiten, stelt eventueel het moreel jaarverslag van de vereniging op en is verantwoordelijk voor de voorlichting van de studenten op vlak van studentenvertegenwoordiging.
        \item De penningmeester houdt de boekhouding van de vereniging bij, stelt een financieel verslag op van elke activiteit, stelt het globaal financieel jaarverslag van de kring op, is verantwoordelijk, samen met de voorzitter, voor alle inkomsten en uitgaven van de kringvereniging.
Cumulatie van hoofdfunctie is enkel mogelijk met de ”Secretaris” functie. Het bestuur kan uitgebreid worden met meerdere leden en functies, zijnde gewone bestuursleden.
      \end{enumerate}
      \item De voorzitter van de vereniging (al dan niet in waarnemende functie) kan niet tezelfdertijd tot het kernbestuur behoren van een bestuurs- of beslissingsorgaan aan de VUB.
      \item Ten minste de 1/3 van het kernbestuur en 1/10 van de gewone bestuursleden moet ingeschreven zijn aan de VUB.
    \end{enumerate}
  \end{artikel}

  \begin{artikel}\hnl
    Het bestuur is belast met de dagelijkse werking van de vereniging.
    Het wordt voorgezeten door de voorzitter van de vereniging.
  \end{artikel}

  \begin{artikel}\hnl
    Het bestuur wordt samengeroepen door de voorzitter van de vereniging.
    Een buitengewone vergadering moet samengeroepen worden op aanvraag van 1/4 van de leden van het bestuur.
  \end{artikel}

  \begin{artikel}\hnl
    Het bestuur kan slechts geldig vergaderen wanneer de helft van het bestuur aanwezig is.
    Beslissingen worden genomen bij gewone meerderheid (dit is het geheel getal onmiddellijk groter dan de helft van de geldig uitgebrachte stemmen).
    Bij staking van stemmen is de stem van de voorzitter doorslaggevend.
  \end{artikel}

  \begin{artikel}\hnl
    De dagorde wordt bepaald door de voorzitter.
    Een lid van het bestuur kan een punt op de dagorde laten plaatsen mits kennisgeving aan de voorzitter ten laatste een 5 dagen voor de vergadering.
  \end{artikel}

  \begin{artikel}\hnl
    De kandidaturen voor bestuursfuncties dienen bij de voorzitter van de vereniging ingediend worden en dit ten laatste 1 uur voor de algemene vergadering.
    De kandidaat voorzitter dient minstens 1 jaar bestuurslid te zijn geweest.
    Indien dit niet mogelijk is of indien dergelijke kandidaat voorzitter niet verkozen wordt, dan pas kan iemand anders in aanmerking komen voor de functie van voorzitter.
  \end{artikel}

  \begin{artikel}\hnl
    Het bestuur wordt jaarlijks verkozen tijdens geheime en schriftelijke verkiezingen.
    De kandidaat bestuursleden worden verkozen bij meerderheid van stemmen.
    De verkiezingen sluiten aan op de algemene vergadering.
  \end{artikel}

  \begin{artikel}\hnl
    Bij bestuursverkiezingen dient minstens 1/4 van de effectieve leden aan de stemming deel te nemen.
    Indien dit quorum niet bereikt wordt dienen nieuwe verkiezingen uitgeschreven te worden en dit totdat het quorum wel wordt behaald.
    Het oud bestuur behoudt zijn mandaat tot het nieuw bestuur verkozen is.
    Elk aanwezig lid heeft één stem.
  \end{artikel}

\end{titel}

% ---------- NIET NALEVEN EN ONTBINDEN ----------

\begin{titel}\ttext{Niet naleving van de statuten en ontbinding van de vereniging}

  \begin{artikel}\hnl
    Elk lid van de vereniging wordt verondersteld de statuten gelezen en goedgekeurd te hebben.
    Bij niet naleving van deze statuten kan er klacht ingediend worden bij de algemene vergadering van de vereniging of het Brussels StudentenGenootschap.
  \end{artikel}

  \begin{artikel}\hnl
    Om tot ontbinding van de kring vereniging over te gaan het akkoord vereist van 4/5 van de aanwezige of vertegenwoordigde effectieve leden.
    De beslissing om tot ontbinding over te gaan kan enkel genomen worden tijdens de algemene vergadering.
    Hiervoor moet er minstens de helft van de effectieve leden aanwezig is.
  \end{artikel}

  \begin{artikel}\hnl
    Bij ontbinding van de vereniging verwittigt het bestuur het Brussels StudentenGenootschap.
    Indien er een procedure is voorgeschreven, al dan niet door de moedervereniging, wordt deze gevolgd.
    Voor alle goederen of vorderingen die verworven zijn met VUB gelden, is de VUB een bevoorrechte schuldeiser.
    In alle andere gevallen is het de algemene vergadering die de dienst Studentenbeleid van de VUB aanduidt als vereffenaar.
    Deze zal desgevallend een curator aanstellen die ervoor zal zorgen dat, na vereffening van alle schulden, de resterende bezittingen van de vereniging aan in voorkomend geval het statutair omschreven doel of het doel dat aangeduid wordt door de algemene vergadering op de ontbindingsvergadering zullen worden overgedragen.
  \end{artikel}


\end{titel}

\end{document}

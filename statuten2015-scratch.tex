% Originele statuten van Het Vrije Podium
\documentclass[a4paper,10pt]{article}
\usepackage[dutch]{babel}
\usepackage{amsthm}
\usepackage{graphicx}
\usepackage[tablegrid]{vhistory}

\theoremstyle{definition}
\newtheorem{titel}{\newline\Large Titel}
\newtheorem{artikelbase}{\large Artikel}
\newenvironment{artikelbis}
  {\addtocounter{artikel}{-1}
   \renewcommand{\theartikel}{\arabic{artikel}.bis}
   \begin{artikel}}
  {\end{artikel}}

\newenvironment{artikel}
  {\begin{artikelbase}}
  {\smallskip
   \end{artikelbase}}

\renewcommand{\familydefault}{\sfdefault}

\newcommand{\ttext}[1]{\Large \textbf{#1} \normalsize}
\newcommand{\ttextcr}{\hfill\newline}
\newcommand{\ttextenum}{\mbox{}}

\title{\vspace{-2.0cm}\includegraphics[scale=0.3]{img/logo.png}\\
  \vspace{1cm}Statuten Het Vrije Podium}
\author{Dennis D'Ooghe\\Voorzitter 2014-2015}
\date{Versie 05/04/2015} % Vul datum in als je dit in de header wilt

\begin{document}
\maketitle

\begin{versionhistory}
  \vhEntry{1.0}{26/03/2015}{Dennis D'Ooghe}{Eerste versie}
  \vhEntry{1.1}{01/04/2015}{Dennis D'Ooghe}{Aanpassingen na open overleg met leden}
  \vhEntry{1.2}{21/04/2015}{Dennis D'Ooghe}{Aanpassingen na verder overleg}
\end{versionhistory}

\newpage

% ---------- DE KRING ----------

\begin{titel}\ttext{Basis van de vereniging}

  \begin{artikel}\ttextcr
    De vereniging draagt de naam ''Het Vrije Podium'', afgekort ''HVP''.
  \end{artikel}

  \begin{artikel}\ttextcr
    De vereniging onderschrijft het principe van het Vrij Onderzoek.
  \end{artikel}

  \begin{artikel}\ttextcr
    De vereniging erkent en onderschrijft de Universele verklaring van de rechten van de mens en het Europees verdrag tot bescherming van de rechten van de mens en de fundamentele vrijheden, ondertekend te Rome op 4 november 1950: Titel I (art 1 tot en met 18)(B.S. 19/08/1955).
  \end{artikel}

  \begin{artikel}\label{kring-oprichting}\ttextcr
    De vereniging is opgericht in september 2014 voor onbepaalde duur.
    De oprichters zijn Dennis D'Ooghe en Valerie Lever.
  \end{artikel}

  \begin{artikel}\ttextcr
    De vereniging is gevestigd op Campus Etterbeek van de Vrije Universiteit Brussel (VUB), Pleinlaan 2, 1050 Elsene.
  \end{artikel}

  \begin{artikel}\ttextcr
    Het doel van de vereniging is het samenbrengen en ondersteunen van creatieve studenten, vooral gericht naar de podium- en theatergerelateerde kunsten.
    Dit door hen op een laagdrempelige manier de mogelijkheid te bieden om deel te nemen aan activiteiten en producties, zowel voor als achter de schermen, en hen een podium te bieden om hun creatieve instincten te ontplooien.\\
    Verder staat de vereniging open voor mogelijkheden om de Nederlandstalige podiumkunsten in Brussel te ondersteunen.
  \end{artikel}

  \begin{artikel}\ttextenum
    \begin{enumerate}
      \item
        Het zwaartepunt van de activiteiten en ledenwerving van de vereniging ligt op de VUB, waarmee de vereniging een sterke band heeft.
      \item
        Daarnaast streeft de vereniging ernaar ook buiten de VUB culturele contacten te leggen en activiteiten te ontplooien.
    \end{enumerate}
  \end{artikel}

  \begin{artikel}\ttextcr
    Ter ondersteuning van zijn doelen, streeft de vereniging onder andere elk academiejaar naar de organisatie van volgende activiteiten waaraan zijn leden kunnen deelnemen.
    \begin{enumerate}
      \item Een publiek toegankelijk optreden op een VUB-campus.
      \item Een publiek toegankelijk optreden buiten de VUB-campussen of de publieke digitale beschikbaarheid van een optreden of project.
      \item Minstens tien samenkomsten of activiteiten die openstaan voor VUB-studenten (met een absoluut minimum van vijf).
    \end{enumerate}
  \end{artikel}

  \begin{artikel}\ttextcr
    De vereniging doet niet aan ideologische of partijpolitieke stellingname, noch in communicatie, noch in zijn activiteiten of producties.
    Dit, zonder afbreuk te doen aan het feit dat theater en andere podiumkunsten een licht moeten kunnen schijnen op sociale, economische en maatschappelijke problemen.
  \end{artikel}

\end{titel}

% --------- DE LEDEN ----------

\begin{titel}\ttext{De leden}

  \begin{artikel}\ttextcr
    De vereniging telt effectieve leden en ereleden.
    \begin{enumerate}
      \item
        Effectieve leden zijn leden die tot de doelgroep van de vereniging behoren en hun lidgeld hebben betaald.
      \item
        Ereleden zijn leden die hun erelidgeld hebben betaald.
      \item
        Een lid kan tegelijk effectief lid en erelid zijn.
    \end{enumerate}
  \end{artikel}

  \begin{artikel}\ttextcr
    De vereniging telt te allen tijde minstens 10 effectieve leden die ingeschreven zijn aan de VUB.
  \end{artikel}

  \begin{artikel}\ttextcr
    Elk lid krijgt van het bestuur een fysiek bewijs van lidmaatschap, dat ook duidelijk aangeeft voor welk type lidmaatschap het geldt.
  \end{artikel}

  \begin{artikel}\ttextcr
    Lidgelden en eventuele ledenvoordelen worden bepaald door het bestuur.
  \end{artikel}

\end{titel}

% --------- DE ALGEMENE VERGADERING ----------

\begin{titel}\ttext{De algemene vergadering}

  \begin{artikel}\ttextcr
    De algemene vergadering is samengesteld uit alle effectieve leden.
    Zij wordt voorgezeten door de voorzitter van de vereniging en komt minstens 1 maal per academiejaar samen.
    De periode tussen 2 opeenvolgende algemene vergaderingen mag maximum 14 maanden duren.
  \end{artikel}

  \begin{artikel}\ttextcr
    De algemene vergadering beslist over de goedkeuring van het jaarverslag van het uittredend bestuur.
    Op de algemene vergadering worden tevens kandidaat-bestuursleden voorgesteld.
    De algemene vergadering is bevoegd de statuten van de vereniging te wijzigen volgens de procedure vermeld in art.~\ref{av-beslissen}
  \end{artikel}

  \begin{artikel}\label{av-samenroeping}\ttextenum
    \begin{enumerate}
      \item
        De algemene vergadering wordt samengeroepen door de voorzitter van de vereniging.
      \item
        Een buitengewone algemene vergadering moet worden samengeroepen na schriftelijke of elektronische aanvraag van 1/2 van de bestuursleden van de vereniging of na schriftelijke of elektronische aanvraag van 1/4 van de effectieve leden van de vereniging.
      \item
        De leden moeten minstens 14 dagen op voorhand verwittigd worden, via schriftelijke of elektronische weg, over de datum, tijd en locatie van de (buitengewone) algemene vergadering.
        Deze samenroeping moet een gedetailleerde dagorde bevatten.
    \end{enumerate}
  \end{artikel}

  \begin{artikel}\label{av-dagorde}\ttextcr
    De algemene vergadering kan slechts geldig vergaderen over punten die in de dagorde voorkomen.
    Elk lid kan een punt op de dagorde laten verschijnen mits dit ten minste 5 dagen voor de vergadering schriftelijk of elektronisch bij de voorzitter van de vereniging wordt ingediend.
  \end{artikel}

  \begin{artikel}\ttextcr
    Indien er punten toegevoegd worden aan de dagorde via de procedure uit art.~\ref{av-dagorde}, dient een ge"updatete dagorde verstuurd te worden naar de leden via schriftelijke of elektronische weg, ten laatste 48 uur voor de start van de algemene vergadering.
  \end{artikel}

  \begin{artikel}\ttextcr
    De algemene vergadering kan slechts geldig vergaderen indien 1/10 van de effectieve leden aanwezig is.
    Indien dit quorum niet bereikt wordt, is de volgende algemene vergadering geldig, ongeacht het aantal aanwezige leden.
    Tussen beide algemene vergaderingen moet minstens een tijdspanne van 15 dagen liggen.
  \end{artikel}

  \begin{artikel}\label{av-beslissen}\ttextcr
    De beslissingen worden genomen bij gewone meerderheid (dit is het geheel getal onmiddellijk groter dan de helft) van de aanwezige stemmen.
    Bij wijziging van de statuten is een 2/3 meerderheid van de aanwezige stemmen vereist en dient ten minste 2/3 van de effectieve leden aanwezig te zijn.
  \end{artikel}

  \begin{artikel}\ttextcr
    Elk stemgerechtigd lid kan zich op de algemene vergadering laten vertegenwoordigen door een ander stemgerechtigd lid via een geschreven en ondertekende verklaring.
    Elk lid kan op dergelijke wijze slechts 1 ander lid vertegenwoordigen.
  \end{artikel}

  \begin{artikel}\ttextcr
    Bij een stemming over de wijziging van de statuten zijn de oprichters (bepaald in art.~\ref{kring-oprichting}) altijd stemgerechtigd.
    In dergelijk geval dienen zij op dezelfde wijze ingelicht te worden als vermeld in art.~\ref{av-samenroeping}\S3.
  \end{artikel}


\end{titel}

% ---------- HET BESTUUR - SAMENSTELLING & WERKING----------

\begin{titel}\ttext{Het bestuur}

  \begin{artikel}Samenstelling\ttextenum
    \begin{enumerate}
      \item Het bestuur telt minstens 5 kernfuncties die samen het kernbestuur vormen:
      \begin{description}
        \item[Voorzitter]
          De voorzitter zit de bestuursvergaderingen en de algemene vergadering voor, organiseert de algemene vergadering, vertegenwoordigt de vereniging op allerhande gelegenheden, organiseert allerhande activiteiten en staat in voor de verkiezing van het nieuw bestuur.
        \item[Ondervoorzitter]
          De ondervoorzitter vervangt de voorzitter bij diens afwezigheid.
        \item[Artistiek directeur]
          De artistiek directeur bewaakt de artistieke en creatieve richting van de vereniging, controleert en is het aanspreekpunt voor eventuele productieleiders (art.~\ref{bestuur-productie}), en zorgt ervoor, samen met de voorzitter, dat er genoeg creatieve vari"etieit is in de georganiseerde activiteiten om te voldoen aan de noden van de leden.
        \item[Secretaris]
          De secretaris maakt een verslag van alle bestuursvergaderingen, van alle activiteiten, stelt het moreel jaarverslag van de vereniging op en is verantwoordelijk voor de voorlichting van de studenten op vlak van studentenvertegenwoordiging.
        \item[Penningmeester]
          De penningmeester houdt de boekhouding van de vereniging bij, stelt een financieel verslag op van elke activiteit, stelt het globaal financieel jaarverslag van de vereniging op, is verantwoordelijk, samen met de voorzitter, voor alle inkomsten en uitgaven van de vereniging.
      \end{description}

      \item
        De ''secretaris''-fuctie en de ''artistiek directeur''-functie zijn de enige kernfuncties die gecumuleerd mogen worden met een andere kernfunctie.
        In geen geval kan iemand meer dan 2 kernfuncties uitoefenen.
      \item Het bestuur kan verder uitgebreid worden met meerdere leden en functies, zijnde gewone bestuursleden en bestuursfuncties.
        Hun functies kunnen gedefinieerd worden in een intern reglement (art.~\ref{varia-reglement}).
    \end{enumerate}
  \end{artikel}

  \begin{artikel}Voorwaarden\ttextenum
    \begin{enumerate}
      \item De voorzitter van de vereniging (al dan niet in waarnemende functie) kan niet tezelfdertijd tot het kernbestuur behoren van een bestuurs- of beslissingsorgaan aan de VUB, of van een door het BSG erkende kring.
      \item Leden van het kernbestuur van de vereniging kunnen niet tezelfdertijd voorzitter (al dan niet in waarnemende functie) zijn van een andere, door het BSG erkende kring. %TODO above may be redundant?
      \item Ten minste 1/3 van het kernbestuur en 1/10 van de gewone bestuursleden moet ingeschreven zijn aan de VUB.
      \item Een lid zonder specifieke (kern)bestuursfunctie kan geen deel uitmaken van het bestuur.
    \end{enumerate}
  \end{artikel}

  \begin{artikel}\ttextcr
    Het bestuur is belast met de dagelijkse werking van de vereniging.
    Het wordt voorgezeten door de voorzitter van de vereniging.
  \end{artikel}

  \begin{artikel}\ttextcr
    Het bestuur wordt samengeroepen door de voorzitter van de vereniging.
    Een buitengewone vergadering moet samengeroepen worden op aanvraag van 1/4 van de leden van het bestuur.
  \end{artikel}

  \begin{artikel}\ttextcr
    Het bestuur kan slechts geldig vergaderen wanneer de helft van het bestuur aanwezig is.
    Beslissingen worden genomen bij gewone meerderheid van de geldig uitgebrachte stemmen.
    Bij staking van stemmen is de stem van de voorzitter doorslaggevend.
  \end{artikel}

  \begin{artikel}\ttextcr
    De dagorde wordt bepaald door de voorzitter.
    Een lid van het bestuur kan een punt op de dagorde laten plaatsen mits kennisgeving aan de voorzitter ten laatste 24 uur voor de vergadering.
    De voorzitter communiceert de dagorde naar alle leden van het bestuur, ten laatste 24 uur voor de vergadering.
  \end{artikel}

\end{titel}

% ---------- HET BESTUUR - VERKIEZING ----------
\begin{titel}\ttext{Toetreding tot het bestuur}

  \begin{artikel}\ttextcr
    Alleen effectieve leden zijn verkiesbaar in de organen van de vereniging.
  \end{artikel}

  \begin{artikel}\ttextenum
    \begin{enumerate}
      \item
        De kandidaturen voor bestuursfuncties dienen bij de voorzitter van de vereniging ingediend worden.
      \item
        Een kandidaat voor een kernbestuursfunctie dient zijn kandidatuur minstens 7 dagen voor de algemene vergadering in te dienen.
      \item
        Een kandidaat voor een andere bestuursfunctie kan zijn kandidatuur indienen tot het begin van de stemming over de desbetreffende functie.
      \item
        De kandidaat-voorzitter dient minstens 1 jaar bestuurslid te zijn geweest.
        Indien dit niet mogelijk is of indien dergelijke kandidaat-voorzitter niet verkozen wordt, dan pas kan iemand anders in aanmerking komen voor de functie van voorzitter.
    \end{enumerate}
  \end{artikel}

  \begin{artikel}\ttextcr
    Het bestuur wordt jaarlijks verkozen tijdens geheime en schriftelijke verkiezingen.
    De kandidaat-bestuursleden worden verkozen bij meerderheid van stemmen.
    De verkiezingen sluiten aan op de algemene vergadering.
  \end{artikel}

  \begin{artikel}\ttextcr
    Bij bestuursverkiezingen dient minstens 1/4 van de effectieve leden aan de stemming deel te nemen.
    Indien dit quorum niet bereikt wordt dienen nieuwe verkiezingen uitgeschreven te worden en dit totdat het quorum wel wordt behaald.
    Het oud bestuur behoudt zijn mandaat tot het nieuw bestuur verkozen is.
    Elk aanwezig of vertegenwoordigd stemgerechtigd lid heeft 1 stem.
  \end{artikel}

  \begin{artikel}\label{bestuur-cooptatie}\ttextenum
    \begin{enumerate}
      \item
        Verkiesbare leden kunnen te allen tijde door het verkozen bestuur geco"opteerd worden in gewone bestuursfuncties (''de functie''), indien het bestuur dit nodig acht om de goede werking van de vereniging te verzekeren.
      \item
        Het lid behoudt de functie in principe tot het moment waarop een nieuw bestuur verkozen wordt.
      \item
        Het bestuur kan op het moment van co"optatie echter beslissen om de functie toe te kennen onder voorwaarden of voor een bepaalde periode.
        In dergelijk geval verliest het lid de functie wanneer niet meer aan de voorwaarden voldaan wordt of de periode verlopen is.
    \end{enumerate}
  \end{artikel}

\end{titel}

% ---------- PRODUCTIELEIDING ----------
\begin{titel}\ttext{Productieleiding}

  \begin{artikel}Definitie\label{bestuur-productie}\ttextenum
    \begin{enumerate}
      \item
        Het bestuur kan de organisatie van grote of wederkerende activiteiten of producties (''de activiteit'') toevertrouwen aan een effectief lid van de vereniging.
        Dit lid krijgt volgens de procedure van art.~\ref{bestuur-cooptatie} de bestuursfunctie van ''productieleider'' van de activiteit, voor de duur van de activiteit en onder voorwaarde van de goede uitoefening ervan.
      \item
        Het bestuur bepaalt de omkadering en duur van de activiteit, daarbinnen krijgt de productieleider autonomie om de activiteit tot een goed einde te brengen.
      \item
        Het bestuur kan beslissen een budget toe te wijzen aan de activiteit, waarover de productieleider autonoom mag beschikken.
        Elk lid van het bestuur kan de rekeningen van de activiteit opvragen ter controle.
      \item
        Niettegenstaande de autonomie van de productieleider, dient deze zich te houden aan de artistieke en creatieve richting van de vereniging.
        Dit wordt gecontroleerd door de artistiek directeur.
        De productieleider heeft daarenboven de plicht de artistiek directeur op de hoogte te houden van de voortgang van de activiteit.
      \item
        TODO: duidelijk de controlerechten van de AD definieren.
    \end{enumerate}
  \end{artikel}

  \begin{artikel}Beperkingen\ttextenum
    \begin{enumerate}
      \item
        Een lid kan voor slechts 1 productie of activiteitenreeks tegelijk aangesteld worden als productieleider.
      \item
        Een kernbestuurslid kan niet aangesteld worden als productieleider, tenzij minstens 3/4 van het bestuur hiermee instemt.
    \end{enumerate}
  \end{artikel}
  
%  \begin{artikel}\label{bestuur-controle}\ttextenum
%    \begin{enumerate}
%      \item
%        Zowel de voorzitter als de artistiek directeur (of diens vervanger volgens art.~\ref{bestuur-vervanging}) kunnen elk autonoom vaststellen dat de productieleider zich niet houdt aan de voorwaarden van art.~\ref{bestuur-productie}.
%        De productieleider mag dan voorlopig zijn functie niet meer uitoefenen.
%      \item
%        In dergelijk geval moet de persoon die deze vaststelling maakt (''de verantwoordelijke'') zo snel mogelijk het bestuur inlichten.
%        De voorzitter roept vervolgens binnen de 7 dagen een bestuursvergadering samen, waarop de verantwoordelijke de vaststelling moet verklaren.
%      \item
%        Het bestuur kan met een 2/3 meerderheid beslissen dat de vaststelling niet correct is en de productieleider in zijn functie herstellen.  
%        In elk ander geval wordt aangenomen dat het lid niet meer voldoet aan de voorwaarden van functie van productieleider, zoals bedoeld in art.~\ref{bestuur-cooptatie}.
%    \end{enumerate}
%  \end{artikel}

  \begin{artikel}\label{bestuur-vervanging}\ttextcr
    Indien de artistiek directeur eveneens productieleider is voor een activiteit, is deze niet in staat zichzelf te controleren zoals bepaald in art.~\ref{bestuur-productie}. % Todo, heel letterlijk slaat dit enkel op §4
    De voorzitter duidt een ander lid van het kernbestuur aan om deze rechten en plichten over te nemen voor de duur van de activiteit en enkel met betrekking tot de activiteit.
    De artistiek directeur behoudt zijn rechten en plichten voor alle andere activiteiten.
  \end{artikel}

\end{titel}

% ---------- NIET NALEVEN EN ONTBINDEN ----------

\begin{titel}\ttext{Niet naleving van de statuten en ontbinding van de vereniging}

  \begin{artikel}\ttextcr
    Elk lid van de vereniging wordt verondersteld de statuten gelezen en goedgekeurd te hebben.
    Bij niet naleving van deze statuten kan er klacht ingediend worden bij de algemene vergadering van de vereniging of het Brussels StudentenGenootschap.
  \end{artikel}

  \begin{artikel}\ttextcr
    Om tot ontbinding van de vereniging over te gaan is het akkoord vereist van 4/5 van de aanwezige of vertegenwoordigde effectieve leden.
    De beslissing om tot ontbinding over te gaan kan enkel genomen worden tijdens de algemene vergadering.
    Hiervoor moet minstens de helft van de effectieve leden aanwezig zijn.
  \end{artikel}

  \begin{artikel}\ttextcr
    Bij ontbinding van de vereniging verwittigt het bestuur het Brussels StudentenGenootschap.
    Voor alle goederen of vorderingen die verworven zijn met VUB gelden, is de VUB een bevoorrechte schuldeiser.
    In alle andere gevallen is het de algemene vergadering die de dienst Studentenbeleid van de VUB aanduidt als vereffenaar.
    Deze zal desgevallend een curator aanstellen die ervoor zal zorgen dat, na vereffening van alle schulden, de resterende bezittingen van de vereniging overgedragen worden aan een of meerdere organisaties die aangeduid worden op de ontbindingsvergadering van de vereniging.
    Deze organisaties moeten het promoten of ondersteunen van de amateurpodiumkunsten in Brussel als een van hun doelen hebben.
  \end{artikel}

\end{titel}

% ---------- VARIA ----------

\begin{titel}\ttext{Andere bepalingen}

  \begin{artikel}\label{varia-reglement}\ttextcr
    Het bestuur en de werking van de vereniging kunnen verder geregeld worden in een intern reglement.
    In dergelijk geval is dit reglement te allen tijde ondergeschikt aan deze statuten en er nooit mee in tegenspraak.
    Het opstellen, goedkeuren en wijzigen van dergelijk reglement is de verantwoordelijkheid van het bestuur van de vereniging.
  \end{artikel}

\end{titel}


\end{document}

% Originele statuten van Het Vrije Podium
\documentclass[a4paper,10pt]{article}
\usepackage[dutch]{babel}
\usepackage{amsthm}
\usepackage{graphicx}

\theoremstyle{definition}
\newtheorem{titel}{\newline\Large Titel}
\newtheorem{artikel}{\large Artikel}
\renewcommand{\familydefault}{\sfdefault}

\newcommand{\ttext}[1]{\Large \textbf{#1} \normalsize}
\newcommand{\ttextcr}{\hfill\newline}
\newcommand{\ttextenum}{\mbox{}}

\title{\vspace{-2.0cm}\includegraphics[scale=0.3]{img/logo.png}\\
  \vspace{1cm}Statuten Het Vrije Podium\\Ontwerpdocument}
\author{Dennis D'Ooghe\\Voorzitter 2014-2015}
\date{Versie 26/03/2015} % Vul datum in als je dit in de header wilt

\begin{document}
\maketitle

% ---------- DE KRING ----------

\begin{titel}\ttext{Basis van de kring}

  \begin{artikel}\ttextcr
    De vereniging draagt de naam ''Het Vrije Podium'', afgekort ''HVP''.
  \end{artikel}

  \begin{artikel}\ttextcr
    De vereniging onderschrijft het principe van het Vrij Onderzoek.
  \end{artikel}

  \begin{artikel}\ttextcr
    De vereniging erkent en onderschrijft de Universele verklaring van de rechten van de mens en het Europees verdrag tot bescherming van de rechten van de mens en de fundamentele vrijheden, ondertekend te Rome op 4 november 1950: Titel I (art 1 tot en met 18)(B.S. 19/08/1955).
  \end{artikel}

  \begin{artikel}\label{kring-oprichting}\ttextcr
    De vereniging is opgericht in september 2014 voor onbepaalde duur.
    De oprichters zijn Dennis D'Ooghe en Valerie Lever.
  \end{artikel}

  \begin{artikel}\ttextcr
    De vereniging is gevestigd op Campus Etterbeek van de Vrije Universiteit Brussel (VUB), Pleinlaan 2, 1050 Elsene.
  \end{artikel}

  \begin{artikel}\ttextcr
        Het doel van de de vereniging is het samenbrengen en ondersteunen van creatieve studenten, vooral gericht naar de podiumkunsten.
        Dit door hen, laagdrempelig, de mogelijkheid te bieden om deel te nemen activiteiten en producties, zowel voor als achter de schermen; en hen een podium bieden om hun creatieve instincten te ontplooien.
        De domeinen waarop de vereniging focust zijn, zonder andere uit te sluiten, o.a.
        \begin{itemize}
          \item Theater
          \item Improvisatie
          \item Online filmproductie
          \item Het cre\"eren van eigen theaterteksten
        \end{itemize}
        Daarnaast promoot de vereniging ook de Nederlandstalige amateurkunsten in Brussel.
  \end{artikel}

  \begin{artikel}\ttextenum
    \begin{enumerate}
      \item Het zwaartepunt van de activiteiten en ledenwerving van de vereniging ligt op de VUB, waarmee de vereniging een sterke band heeft.
      \item
        Daarnaast streeft de vereniging ernaar ook buiten de VUB culturele contacten te leggen en activiteiten te ontplooien.
        Dit kan bijvoorbeeld in de vorm van optredens of workshops.
    \end{enumerate}
  \end{artikel}

  \begin{artikel}\ttextcr
    Ter ondersteuning van zijn doelen, streeft de vereniging elk academiejaar minstens naar de organisatie van volgende activiteiten waaraan zijn leden deelnemen.
    \begin{enumerate}
      \item Een publiek toegankelijk optreden op een VUB-campus.
      \item Een publiek toegankelijk optreden buiten de VUB-campussen of de publieke digitale beschikbaarheid van een optreden.
      \item Tien avondlijke ledensamenkomsten op een VUB-campus.
    \end{enumerate}
  \end{artikel}

  \begin{artikel}\ttextcr
    De vereniging doet niet aan politieke stellingname, noch in communicatie, noch in zijn activiteiten of producties.
    Dit, zonder afbreuk te doen aan het feit dat theater en andere podiumkunsten een licht moeten kunnen schijnen op sociale, economische en maatschappelijke problemen.
  \end{artikel}

\end{titel}

% --------- DE LEDEN ----------

\begin{titel}\ttext{De leden}

  \begin{artikel}\ttextcr
    De vereniging telt effectieve leden en ereleden.
    \begin{enumerate}
      \item
        Effectieve leden zijn leden die tot de doelgroep van de vereniging behoren en hun lidgeld hebben betaald.
      \item
        Ereleden zijn leden die hun erelidgeld betaald hebben.
      \item
        Een lid kan tegelijk effectief lid en erelid zijn.
    \end{enumerate}
  \end{artikel}

  \begin{artikel}\ttextcr
    De vereniging telt te allen tijde minstens 10 effectieve leden die ingeschreven zijn aan de VUB.
  \end{artikel}

\end{titel}

% --------- DE ALGEMENE VERGADERING ----------

\begin{titel}\ttext{De algemene vergadering}

  \begin{artikel}\ttextcr
    De algemene vergadering is samengesteld uit alle effectieve leden.
    Zij wordt voorgezeten door de voorzitter van de vereniging en komt minstens 1 maal per jaar samen.
  \end{artikel}

  \begin{artikel}\ttextcr
    De algemene vergadering beslist over de goedkeuring van het jaarverslag van het uittredend bestuur.
    Op de algemene vergadering worden tevens kandidaten bestuursleden voorgesteld.
    De algemene vergadering is bevoegd de statuten van de vereniging te wijzigen volgens de procedure vermeld in art.\ref{av-beslissen}
  \end{artikel}

  \begin{artikel}\label{av-bijeenroeping}\ttextcr
    De algemene vergadering wordt samengeroepen door de voorzitter van de vereniging.
    Een buitengewone algemene vergadering moet worden samengeroepen na schriftelijke aanvraag van 1/2 van de bestuursleden van de vereniging of na schriftelijke aanvraag van 1/4 van de effectieve leden van de vereniging.
    De leden moeten minstens 14 dagen op voorhand verwittigd worden via elektronische weg.
    Deze bijeenroeping moet een gedetailleerde dagorde bevatten.
  \end{artikel}

  \begin{artikel}\label{av-dagorde}\ttextcr
    De algemene vergadering kan slechts geldig vergaderen over punten die in de dagorde voorkomen.
    Elk lid kan een punt op de dagorde laten verschijnen mits dit ten minste 5 dagen voor de vergadering schriftelijk of elektronisch bij de voorzitter van de vereniging wordt ingediend.
  \end{artikel}

  \begin{artikel}\ttextcr
    Indien er punten toegevoegd worden aan de dagorde via de procedure uit art.\ref{av-dagorde}, dient een ge\"updatete dagorde verstuurd te worden naar de leden via elektronische weg, ten laatste 36 uur voor de start van de algemene vergadering.
  \end{artikel}

  \begin{artikel}\ttextcr
    De algemene vergadering kan slechts geldig vergaderen indien 1/10 van de effectieve leden aanwezig is.
    Indien dit quorum niet bereikt wordt, is de volgende algemene vergadering geldig, ongeacht het aantal aanwezige leden.
    Tussen beide algemene vergaderingen moet minstens een tijdspanne van 15 dagen liggen.
  \end{artikel}

  \begin{artikel}\label{av-beslissen}\ttextcr
    De beslissingen worden genomen bij gewone meerderheid van de aanwezige stemmen.
    Bij wijziging van de statuten is een 2/3 meerderheid van de aanwezige stemmen vereist en dient ten minste 2/3 van de effectieve leden aanwezig te zijn.
  \end{artikel}

  \begin{artikel}\ttextcr
    Elk stemgerechtigd lid kan zich op de algemene vergadering laten vertegenwoordigen door een ander stemgerechtigd lid via een geschreven en ondertekende verklaring.
    Elk lid kan op dergelijke wijze slechts 1 ander lid vertegenwoordigen.
  \end{artikel}

  \begin{artikel}\ttextcr
    Bij een stemming over de wijziging van de statuten zijn de oprichters (bepaald in art.\ref{kring-oprichting}) altijd stemgerechtigd.
%    Zij dragen niet bij tot het aanwezigheidsquorum van 2/3 van de effectieve leden uit art.\ref{av-beslissen}, tenzij ze ook effectief lid zijn.
  \end{artikel}


\end{titel}

% ---------- HET BESTUUR - SAMENSTELLING & WERKING----------

\begin{titel}\ttext{Het bestuur}

  \begin{artikel}\ttextenum
    \begin{enumerate}
      \item Het bestuur telt minstens 5 kernfuncties die samen het kernbestuur vormen:
      \begin{description}
        \item[Voorzitter]
          De voorzitter zit de bestuursvergaderingen en de algemene vergadering voor, organiseert de algemene vergadering, vertegenwoordigt de vereniging op allerhande gelegenheden, organiseert allerhande activiteiten en staat in voor de verkiezing van het nieuw bestuur.
        \item[Ondervoorzitter]
          De ondervoorzitter vervangt de voorzitter bij diens afwezigheid.
        \item[Secretaris]
          De secretaris maakt een verslag van alle bestuursvergaderingen, van alle activiteiten, stelt het moreel jaarverslag van de vereniging op en is verantwoordelijk voor de voorlichting van de studenten op vlak van studentenvertegenwoordiging.
        \item[Penningmeester]
          De penningmeester houdt de boekhouding van de vereniging bij, stelt een financieel verslag op van elke activiteit, stelt het globaal financieel jaarverslag van de kring op, is verantwoordelijk, samen met de voorzitter, voor alle inkomsten en uitgaven van de kringvereniging.
        \item[Artistiek directeur]
          De artistiek directeur bewaakt de artistieke en creatieve richting van de vereniging, is het aanspreekpunt voor eventuele productieleiders (art.\ref{bestuur-productie}) en zorgt ervoor, samen met de voorzitter, dat er genoeg creatieve vari\"etieit is in de georganiseerde activiteiten om te voldoen aan de noden van de leden.
      \end{description}

      Cumulatie van kernfunctie is enkel mogelijk met de ''secretaris''- en de ''artistiek directeur''-functie.
      In geen geval kan iemand meer dan 2 kernfuncties uitoefenen.
      \item De voorzitter van de vereniging (al dan niet in waarnemende functie) kan niet tezelfdertijd tot het kernbestuur behoren van een bestuurs- of beslissingsorgaan aan de VUB.
      \item Ten minste de 1/3 van het kernbestuur en 1/10 van de gewone bestuursleden moet ingeschreven zijn aan de VUB.
      \item Het bestuur kan verder uitgebreid worden met meerdere leden en functies, zijnde gewone bestuursleden en bestuursfuncties.
        Hun functies kunnen gedefinieerd worden in een intern reglement (zie ook art.\ref{varia-reglement}).
    \end{enumerate}
  \end{artikel}

  \begin{artikel}\ttextcr
    Het bestuur is belast met de dagelijkse werking van de vereniging.
    Het wordt voorgezeten door de voorzitter van de vereniging.
  \end{artikel}

  \begin{artikel}\ttextcr
    Het bestuur wordt samengeroepen door de voorzitter van de vereniging.
    Een buitengewone vergadering moet samengeroepen worden op aanvraag van 1/4 van de leden van het bestuur.
  \end{artikel}

  \begin{artikel}\ttextcr
    Het bestuur kan slechts geldig vergaderen wanneer de helft van het bestuur aanwezig is.
    Beslissingen worden genomen bij gewone meerderheid (dit is het geheel getal onmiddellijk groter dan de helft van de geldig uitgebrachte stemmen).
    Bij staking van stemmen is de stem van de voorzitter doorslaggevend.
  \end{artikel}

  \begin{artikel}\ttextcr
    De dagorde wordt bepaald door de voorzitter.
    Een lid van het bestuur kan een punt op de dagorde laten plaatsen mits kennisgeving aan de voorzitter ten laatste 24 uur voor de vergadering.
    De voorzitter communiceert de dagorde naar alle leden van het bestuur, ten laatste 24 uur voor de vergadering.
  \end{artikel}

  \begin{artikel}\label{bestuur-productie}\ttextenum
    \begin{enumerate}
      \item
        Het bestuur kan de organisatie van grote of wederkerende activiteiten of producties toevertrouwen aan een effectief lid van de vereniging.
        Dit lid wordt dan aanzien als ''productieleider'' van deze activiteit en wordt in dergelijke hoedanigheid toegevoegd aan het bestuur van de vereniging als gewoon bestuurslid.
      \item
        Het bestuur bepaalt de omlijning van de activiteit, daarbinnen krijgt de productieleider autonomie om de activiteit tot een goed einde te brengen.
        Het bestuur kan beslissen een budget toe te wijzen aan de activiteit, waarover de productieleider eveneens autonoom mag beschikken.
        Elk lid van het bestuur kan de rekeningen van de activiteiten opvragen ter controle.
      \item
        Niettegenstaande deze autonomie, dient de productieleider zich te houden aan de artistieke en creatieve richting van de vereniging.
        Dit wordt gecontroleerd door de artistiek directeur.
      \item
        Een lid kan voor slechts 1 productie of activiteitreeks tegelijk aangesteld worden als productieleider.
      \item
        Een kernbestuurslid kan niet aangesteld worden als productieleider, tenzij minstens 3/4 van het bestuur hiermee instemt.
    \end{enumerate}
  \end{artikel}
    
\end{titel}

% ---------- HET BESTUUR - VERKIEZING ----------

\begin{titel}\ttext{De bestuursverkiezing}

  \begin{artikel}\ttextcr
    Alleen effectieve leden zijn verkiesbaar in de organen van de vereniging.
  \end{artikel}

  \begin{artikel}\ttextenum
    \begin{enumerate}
      \item
        De kandidaturen voor bestuursfuncties dienen bij de voorzitter van de vereniging ingediend worden.
      \item
        Een kandidaat voor een kernbestuursfunctie dient zijn kandidatuur minstens 1 week voor de algemene vergadering in te dienen.
      \item
        Een kandidaat voor een andere bestuursfunctie kan zijn kandidatuur indienen tot het begin van de stemming over de desbetreffende functie.
      \item
        De kandidaat-voorzitter dient minstens 1 jaar bestuurslid te zijn geweest.
        Indien dit niet mogelijk is of indien dergelijke kandidaat voorzitter niet verkozen wordt, dan pas kan iemand anders in aanmerking komen voor de functie van voorzitter.
    \end{enumerate}
  \end{artikel}

  \begin{artikel}\ttextcr
    Het bestuur wordt jaarlijks verkozen tijdens geheime en schriftelijke verkiezingen.
    De kandidaat bestuursleden worden verkozen bij meerderheid van stemmen.
    De verkiezingen sluiten aan op de algemene vergadering.
  \end{artikel}

  \begin{artikel}\ttextcr
    Verkiesbare leden kunnen doorheen het academiejaar door het verkozen bestuur gecoöpteerd worden als gewoon bestuurslid, indien het bestuur dit nodig acht om de goede werking van de vereniging te verzekeren.
  \end{artikel}

  \begin{artikel}\ttextcr
    Bij bestuursverkiezingen dient minstens 1/4 van de effectieve leden aan de stemming deel te nemen.
    Indien dit quorum niet bereikt wordt dienen nieuwe verkiezingen uitgeschreven te worden en dit totdat het quorum wel wordt behaald.
    Het oud bestuur behoudt zijn mandaat tot het nieuw bestuur verkozen is.
    Elk aanwezig of vertegenwoordigd stemgerechtigd lid heeft 1 stem.
  \end{artikel}

\end{titel}

% ---------- NIET NALEVEN EN ONTBINDEN ----------

\begin{titel}\ttext{Niet naleving van de statuten en ontbinding van de vereniging}

  \begin{artikel}\ttextcr
    Elk lid van de vereniging wordt verondersteld de statuten gelezen en goedgekeurd te hebben.
    Bij niet naleving van deze statuten kan er klacht ingediend worden bij de algemene vergadering van de vereniging of het Brussels StudentenGenootschap.
  \end{artikel}

  \begin{artikel}\ttextcr
    Om tot ontbinding van de kring vereniging over te gaan het akkoord vereist van 4/5 van de aanwezige of vertegenwoordigde effectieve leden.
    De beslissing om tot ontbinding over te gaan kan enkel genomen worden tijdens de algemene vergadering.
    Hiervoor moet er minstens de helft van de effectieve leden aanwezig is.
  \end{artikel}

  \begin{artikel}\ttextcr
    Bij ontbinding van de vereniging verwittigt het bestuur het Brussels StudentenGenootschap.
    Voor alle goederen of vorderingen die verworven zijn met VUB gelden, is de VUB een bevoorrechte schuldeiser.
    In alle andere gevallen is het de algemene vergadering die de dienst Studentenbeleid van de VUB aanduidt als vereffenaar.
    Deze zal desgevallend een curator aanstellen die ervoor zal zorgen dat, na vereffening van alle schulden, de resterende bezittingen van de vereniging overgedragen worden aan een of meerdere organisaties die aangeduid worden op de ontbindingsvergadering van de vereniging.
    Deze organisaties moeten het promoten of ondersteunen van de amateurpodiumkunsten in Brussel als een van hun doelen hebben.
  \end{artikel}

\end{titel}

% ---------- VARIA ----------

\begin{titel}\ttext{Andere bepalingen}

  \begin{artikel}\label{varia-reglement}\ttextcr
    Het bestuur en de werking van de vereniging kunnen verder geregeld worden in een intern reglement.
    In dergelijk geval is dit reglement te allen tijde ondergeschikt aan deze statuten en er nooit mee in tegenspraak.
    Het opstellen, goedkeuren en wijzigen van dergelijk reglement is de verantwoordelijkheid van het bestuur van de vereniging.
  \end{artikel}

\end{titel}


\end{document}
